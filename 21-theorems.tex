% Provides: Definitions and styling for theorem-like environments
% Requires: the math-header for \newtheorem, one of the language-headers

\iftoggle{20-math}{

	% shortcut for defining both numbered and unnumbered versions of an environment in a signle command
	% Usage: \newoptionalnumberedtheroem{envName}{counterLike}{Display Name}
	\NewDocumentCommand{\newoptionalnumberedtheorem}{ m m m }{
		\newtheorem{#1}		[#2]{#3}
		\newtheorem*{#1*}		{#3}
	}

	\newtoggle{framedTheorems}
	\settoggle{framedTheorems}{false}

	% theorems need to be defined even when they are not visually surrounded
	\nottoggle{framedTheorems}{

		\theoremstyle{definition}

		% the definition counter by which all other counted environments will be counted
		% needs to be defined by a different syntax -> cannot use \newoptionalnumberedtheorem{…}
		\newtheorem{definition}		{Definition} [section]
		\newtheorem*{definition*}	{Definition}

		\newoptionalnumberedtheorem{proposition}	{definition}	{Proposition}
		\newoptionalnumberedtheorem{lemma}			{definition}	{Lemma}
		\newoptionalnumberedtheorem{axiom}			{definition}	{Axiom}

		\iftoggle{02-language-english}{
			\newoptionalnumberedtheorem	{theorem}		{definition}	{Theorem}
			\newoptionalnumberedtheorem	{corollary}		{definition}	{Corollary}
		}{}
		\iftoggle{02-language-german}{
			\newoptionalnumberedtheorem	{theorem}		{definition}	{Satz}
			\newoptionalnumberedtheorem	{corollary}		{definition}	{Korollar}
		}{}

	}{
		% Frame the environments Definition, Proposition, Lemma, Axiom, Theorem, Corollary, and Example

		% For frames around theorems
		% Loads xcolor if it has not previously been loaded
		\usepackage{mdframed}

		% Default values for all frames
		\mdfsetup{
			skipabove=\baselineskip,
			skipbelow=\baselineskip,
		}

		% styling for almost all frames
		\mdfdefinestyle{theoremstyle}{
			roundcorner=2pt,
			linewidth=0.6pt,
			frametitlerule=true,%
			frametitleaboveskip=7pt,
			frametitlebelowskip=7pt,
			frametitlerulewidth=0.4pt,%
			innertopmargin=\topskip,
			innerbottommargin=\topskip,
			nobreak=true,
			%	frametitlebackgroundcolor=gray!30,
		}

		% Same for both languages
		% Definitions define the numbering	name of nev		printed name	numbering resets with
		\mdtheorem[style=theoremstyle]		{definition}	{Definition}	[section]
		% Everything else follows it		name of env		numbered like	printed name
		\mdtheorem[style=theoremstyle]		{proposition}	[definition]	{Proposition}
		\mdtheorem[style=theoremstyle]		{lemma}			[definition]	{Lemma}
		\mdtheorem[style=theoremstyle]		{axiom}			[definition]	{Axiom}

		\iftoggle{02-language-english}{
			\mdtheorem[style=theoremstyle]	{theorem}		[definition]	{Theorem}
			\mdtheorem[style=theoremstyle]	{corollary}		[definition]	{Corollary}
		}{}
		\iftoggle{02-language-german}{
			\mdtheorem[style=theoremstyle]	{theorem}		[definition]	{Satz}
			\mdtheorem[style=theoremstyle]	{corollary}		[definition]	{Korollar}
		}{}
	}

	% \proofname is language-aware (but I don't know whether babel or some ams-package sets it)
	\NewDocumentEnvironment {subproof} { O{\proofname} +b}{%
		\renewcommand{\qedsymbol}{$\blacksquare$}%
		\begin{proof}[#1]%
			#2
		\end{proof}%
	}{}

	% In-text environments, need to be defined no matter what. 
	% They all use the ‘remark’-theorem-style, i.e. italic theorem name.
	\theoremstyle{remark}
	\newoptionalnumberedtheorem{problem}			{definition}{Problem}
	\newoptionalnumberedtheorem{notation}			{definition}{Notation}

	\iftoggle{02-language-english}{
		\newoptionalnumberedtheorem{terminology}	{definition}{Terminology}
		\newoptionalnumberedtheorem{remark}			{definition}{Remark}
		\newoptionalnumberedtheorem{example}		{definition}{Example}
	}{}
	\iftoggle{02-language-german}{
		\newoptionalnumberedtheorem{terminology}	{definition}{Terminologie}
		\newoptionalnumberedtheorem{remark}			{definition}{Anmerkung}
		\newoptionalnumberedtheorem{example}		{definition}{Beispiel}
	}{}

	\iftoggle{framedTheorems}{
		% examples have a lightgrey background and a darkgray left-line
		\mdfdefinestyle{examplestyle}{%
			topline=false,
			bottomline=false,
			rightline=false,
			linewidth=4pt,
			linecolor=lightgray!70,
			backgroundcolor=lightgray!40,
			innertopmargin=7pt,
			innerbottommargin=7pt,
		}
		\surroundwithmdframed[style=examplestyle]{example}
		\surroundwithmdframed[style=examplestyle]{example*}
	}{}


	\toggletrue{21-theorems}
}{}
