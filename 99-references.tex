% Provides: Styling and names for in-document references such as theorems.
% Requires: /

\usepackage{hyperref}
% Yes, hyperref, you do nothing in draft mode, it’s ok.
\WarningFilter[default-disabled]{hyperref}{Draft mode}
\hypersetup{
	unicode=true,
	colorlinks=true, % style links by coloring them (instead of boxes)
	pdfstartview=FitV,
}
\urlstyle{same}

\iftoggle{is-beamer}{}{
	% Automatically add the type of whatever is being referenced by ‘\cref{…}’ or ‘\Cref{…}’, e.g., ‘Figure 1’
	% Not available in beamer class
	\usepackage{cleveref}

	\iftoggle{21-theorems}{
		% teaches cleverref how to name links to objects of a certain type
		\iftoggle{02-language-english}{
			% Define everything lowercase. \Cref can still be used to achieve upper case.
			\crefname{definition}{definition}{definitions}
			\crefname{proposition}{proposition}{propositions}
			\crefname{lemma}{lemma}{lemmas}
			\crefname{axiom}{axiom}{axioms}

			\crefname{theorem}{theorem}{theorems}
			\crefname{corollary}{corollary}{corollaries}

			\crefname{problem}{problem}{problems}
			\crefname{notation}{notation}{notations}
			\crefname{terminology}{terminology}{terminologies}
			\crefname{remark}{remark}{remarks}
			\crefname{example}{example}{examples}

			\crefname{proof}{proof}{proofs}
		}{}

		\iftoggle{02-language-german}{
			\crefname{definition}{Definition}{Definitionen}
			\crefname{proposition}{Proposition}{Propositionen}
			\crefname{lemma}{Lemma}{Lemmata}
			\crefname{axiom}{Axiom}{Axiome}

			\crefname{theorem}{Satz}{Sätze}
			\crefname{corollary}{Korollar}{Korollare}

			\crefname{problem}{Problem}{Probleme}
			\crefname{notation}{Notation}{Notationen}
			\crefname{terminology}{Terminologie}{Terminologien}
			\crefname{remark}{Anmerkung}{Anmerkungen}
			\crefname{example}{Beispiel}{Beispiele}

			\crefname{proof}{Beweis}{Beweise}
		}{}

		% By default cleveref determines a links type by looking at the counter.
		% However, all my mdframed and newtheorem environments use the same counter ‘definition’.
		% The following introduces a new type for each environment (how they are spelled out is handled above).
		\let\oldproposition\proposition
		\renewcommand{\proposition}{%
			\crefalias{definition}{proposition}%
			\oldproposition}

		\let\oldlemma\lemma
		\renewcommand{\lemma}{%
			\crefalias{definition}{lemma}
			\oldlemma}

		\let\oldaxiom\axiom
		\renewcommand{\axiom}{%
			\crefalias{definition}{axiom}
			\oldaxiom}

		\let\oldtheorem\theorem
		\renewcommand{\theorem}{%
			\crefalias{definition}{theorem}
			\oldtheorem}

		\let\oldcorollary\corollary
		\renewcommand{\corollary}{%
			\crefalias{definition}{corollary}
			\aoldcorollary}

		\let\oldproblem\problem
		\renewcommand{\problem}{%
			\crefalias{definition}{problem}
			\oldproblem}

		\let\oldnotation\notation
		\renewcommand{\notation}{%
			\crefalias{definition}{notation}
			\oldnotation}

		\let\oldterminology\terminology
		\renewcommand{\terminology}{%
			\crefalias{definition}{terminology}
			\oldterminology}

		\let\oldremark\remark
		\renewcommand{\remark}{%
			\crefalias{definition}{remark}
			\oldremark}

		\let\oldexample\example
		\renewcommand{\example}{%
			\crefalias{definition}{example}
			\oldexample}
	}{}
}


\toggletrue{99-references}
