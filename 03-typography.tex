% Provides: Font styling and spacing
% Requires: /

% Font selection and acces to system fonts
\usepackage[no-math]{fontspec}

% Provides the ‘\xspace’ command, which conditionally inserts a space, depending on the following symbol.
% Needed by some other headers.
\usepackage{xspace}

% Similar to ‘xspace’, provides ‘\xcomma’, ‘\xperiod’ and ‘\xperiodcomma’.
\usepackage{xpunctuate}
% Always follow i.e. and e.g. by a comma (Chicago Manual of Style says so)
\iftoggle{02-language-english}{
	\newcommand\eg{e.g.\xcomma}
	\newcommand\ie{i.e.\xcomma}
	\newcommand\wrt{w.r.t.\ }
}{}

% Separates paragraphs by a vertical space instead of (or in addition to) indenting them.
% To keep the indentation of new paragraphs, supply the ‘indent’ option.
% In KOMA-class documents the `parskip`-option should be used
\iftoggle{is-koma-class}{}{
	\usepackage{parskip}
}
% For ‘inspirational’ quotes at the start of chapter or section.
\usepackage{epigraph}

% Dummy text (lorem ipsum …)
\usepackage{lipsum}

% Some packages misbehave with beamer.
\iftoggle{is-beamer}{}{
	% Very slightly adjusts spacing
	\usepackage{microtype}

	% Slightly more spacing between lines
	\usepackage{setspace}
	\setstretch{1.1}

	% Prevents widows and orphans.
	% At least three lines before a page break and at least 3 after it sound fine.
	% The ‘all’-option applies it to the whole document.
	\usepackage[defaultlines=3, all]{nowidow}
}

\toggletrue{03-typography}
