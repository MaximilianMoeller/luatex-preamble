% Provides: Helps during writing
% Requires: /

% ~~ Convention ~~
% The standard way to enable draft-mode is to give the `draft`-option to the document class (i.e., defining it globally)
% and many packages support this option.
% Likewise, the final version of a document is obtained by setting the `final` option globally.
% A problem arrises if neither of these options is given.
% The `final` version should be the default, however, not all packages respect this convention.
% This preamble therefore enforces the `final`-mode in this scenario, but encourages its users to be explicit in the global options.

% defines the `\ifdraft{…}{…}`-command
% Conveniently, its semantics coincides with the above convention.
\usepackage{ifdraft}

\iftoggle{is-koma-class}{
	% Do not show rulers in draft mode for KOMA-classes
	\usepackage[draft=false]{scrlayer-scrpage}
}{}

% The `obeyDraft`-option makes todonotes behave according to the above convention.
\usepackage[obeyDraft, tickmarkheight=1ex,textsize=footnotesize]{todonotes}
% Warning message is bugged (i.e., appears even when marginparwidth is set to more than 2cm)
% it is thus better to ignore the message and leave the formatting of the margin to the documentclass
\WarningFilter[bugged]{todonotes}{The length marginparwidth is less than 2cm}

% For all packages that do not follow the above convention
\ifdraft{
	% Shows the label given by a `\label{…}`-command besides the labeled entity (in the margin)
	\usepackage[inline]{showlabels}
}{}


\toggletrue{19-draft-mode}
