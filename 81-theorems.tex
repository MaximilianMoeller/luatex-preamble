% Provides: Definitions and styling for theorem-like environments
% Requires: the math-header for \newtheorem, one of the language-headers

\iftoggle{80-math}{

	% For frames around theorems
	% Loads xcolor if it has not previously been loaded
	\usepackage{mdframed}

	% Default values for all frames
	\mdfsetup{
		skipabove=\baselineskip,
		skipbelow=\baselineskip,
	}

	% styling for almost all frames
	\mdfdefinestyle{theoremstyle}{
		roundcorner=2pt,
		linewidth=0.6pt,
		frametitlerule=true,%
		frametitleaboveskip=7pt,
		frametitlebelowskip=7pt,
		frametitlerulewidth=0.4pt,%
		innertopmargin=\topskip,
		innerbottommargin=\topskip,
		nobreak=true,
		%	frametitlebackgroundcolor=gray!30,
	}

	% Same for both languages
	% Definitions define the numbering	name of nev		printed name	numbering resets with
	\mdtheorem[style=theoremstyle]		{definition}	{Definition}	[section]
	% Everything else follows it		name of env		numbered like	printed name
	\mdtheorem[style=theoremstyle]		{proposition}	[definition]	{Proposition}
	\mdtheorem[style=theoremstyle]		{lemma}			[definition]	{Lemma}
	\mdtheorem[style=theoremstyle]		{axiom}			[definition]	{Axiom}

	\iftoggle{02-language-english}{
		\mdtheorem[style=theoremstyle]	{theorem}		[definition]	{Theorem}
		\mdtheorem[style=theoremstyle]	{corollary}		[definition]	{Corollary}
	}{}
	\iftoggle{02-language-german}{
		\mdtheorem[style=theoremstyle]	{theorem}		[definition]	{Satz}
		\mdtheorem[style=theoremstyle]	{corollary}		[definition]	{Korollar}
	}{}

	% \proofname is language-aware (but I don't know whether babel or some ams-package sets it)
	\NewDocumentEnvironment {subproof} { O{\proofname} +b}{%
		\renewcommand{\qedsymbol}{$\blacksquare$}%
		\begin{proof}[#1]%
			#2
		\end{proof}%
	}{}

	% some in-text environments, that all use the ‘remark’-theorem-style, i.e. italic theorem name.
	% unfortunately, they have to be defined separately for both numbered and unnumbered use
	\theoremstyle{remark}
	\newtheorem{problem}			[definition]{Problem}
	\newtheorem*{problem*}						{Problem}
	\newtheorem{notation}			[definition]{Notation}
	\newtheorem*{notation*}						{Notation}

	\iftoggle{02-language-english}{
		\newtheorem{terminology}	[definition]{Terminology}
		\newtheorem*{terminology*}				{Terminology}
		\newtheorem{remark}			[definition]{Remark}
		\newtheorem*{remark*}					{Remark}
		\newtheorem{example}		[definition]{Example}
		\newtheorem*{example*}					{Example}
	}{}
	\iftoggle{02-language-german}{
		\newtheorem{terminology}	[definition]{Terminologie}
		\newtheorem*{terminology*}				{Terminologie}
		\newtheorem{remark}			[definition]{Anmerkung}
		\newtheorem*{remark*}					{Anmerkung}
		\newtheorem{example}		[definition]{Beispiel}
		\newtheorem*{example*}					{Beispiel}
	}{}

	% examples have a lightgrey background and a darkgray left-line
	\mdfdefinestyle{examplestyle}{%
		topline=false,
		bottomline=false,
		rightline=false,
		linewidth=4pt,
		linecolor=lightgray!70,
		backgroundcolor=lightgray!40,
		innertopmargin=7pt,
		innerbottommargin=7pt,
	}
	\surroundwithmdframed[style=examplestyle]{example}
	\surroundwithmdframed[style=examplestyle]{example*}


	\toggletrue{81-theorems}
}{}
