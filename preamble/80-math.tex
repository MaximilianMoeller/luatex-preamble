% Provides: Mathematical packages, symbols, notation and shortcuts.
% You will almost always need this, because, why would you write something else than math. ;-)
% Requires: xspace

%% Packages

% For ‘DeclareMathOperator’-command, various alignment-environments (e.g. align, split, gather …),
% better spacing around equality-signs, and many more. In short: indispensible.
\usepackage{amsmath}

% For better theorem-like-environments. Also defines a proof-environment.
% Must be loaded after amsmath.
\usepackage{amsthm}

% Many more math symbols like \nsubseteq, \mapsto, \nexists, \hat{…} and many more.
% Also the \mathbb{…} command for ‘blackboard boald’ character used for sets of numbers.
% Implicitly loads amsfonts, “An extended set of fonts for use in mathematics […]”.
\usepackage{amssymb}

% Some more typesetting like ‘DeclarePairedDelimiter’, ‘\begin{cases}’,‘\overbrace{}’, ‘\mathclap{}’, …
% (the latter is useful for reducing whitespaces that can occur in subscripts or exponents.)
\usepackage{mathtools}

% Provides a good default for the ‘\mathscr{…}’ font style.
% TODO: Maybe customize with the ‘mathalpha’-package later.
\usepackage{mathrsfs}

% TODO: Add description about what this exactly does.
% TODO: are there options to: 
% 					- disable the warnings that this overwrites some ams-things
% 					- make it overwrite all it can
\usepackage{unicode-math}

%% Numbers
\NewDocumentCommand{\N}{}{\ensuremath{\mathbb{N}}\xspace}
\NewDocumentCommand{\Z}{}{\ensuremath{\mathbb{Z}}\xspace}
\NewDocumentCommand{\Q}{}{\ensuremath{\mathbb{Q}}\xspace}
\NewDocumentCommand{\R}{}{\ensuremath{\mathbb{R}}\xspace}
\NewDocumentCommand{\C}{}{\ensuremath{\mathbb{C}}\xspace}

% Funktionen und variablen aus mehreren Buchstaben
% use this for multi-letter variables/
\newcommand{\var}[1]{\ensuremath{\mathrm{#1}}}
\newcommand{\func}[1]{\operatorname{#1}}

\newcommand\questioneq{\stackrel{\text{?}}{=}}

% MathOperator with a ‘*’ allows for ‘lim’-like notation with limits below in display math mode.
% \max, \min, \sup, \inf are already pre-defined by amsmath
\DeclareMathOperator*{\argmax}{arg\,max}
\DeclareMathOperator*{\argmin}{arg\,min}

\newcommand{\quantifier}{\style{display: inline-block; transform: scale(-1, 1)}{\mathsf{Q}}}
\newcommand{\blank}{\not{b}}

% File has been loaded.
\toggletrue{80-math}
