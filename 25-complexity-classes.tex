% Provides: Common style for computational complexity classes and definitions for the most common ones
% Requires: xspace
% Copyright: I adapted most of the complexity classes from Markus Krötzsch’s Lecture on Theoretical Computer Science
% 		- (C) Markus Krötzsch, https://iccl.inf.tu-dresden.de/web/TheoLog2017, CC BY 3.0 DE, and
%		- (C) Maximilian Moeller, https://github.com/MaximilianMoeller/luatex-preamble, CC BY 4.0 Deed 
\hyphenation{Exp-Time} % prevents “Ex-pTime”
\hyphenation{NExp-Time} % prevents “NEx-pTime”
\hyphenation{Exp-Space} % prevents “Ex-pSpace”
\hyphenation{kExp-Time} % prevents “kEx-pTime” and “k-ExpTime”
\hyphenation{NkExp-Time} % prevents “NkEx-pTime” and “N-kExpTime”
\hyphenation{kExp-Space} % prevents “kEx-pSpace” and “k-ExpSpace”

\DeclareMathAlphabet{\mathsc}{OT1}{cmr}{m}{sc}

% Define a common style for complexity classes.
% Also use this for less common classes inside the main document.
\NewDocumentCommand{\complclass}{ m }{\ensuremath{\mathsc{#1}}\xspace}

\NewDocumentCommand{\APX}			{}{\complclass{APX}}
\NewDocumentCommand{\ACzero}		{}{\complclass{AC$_0$}}
\NewDocumentCommand{\LogSpace}		{}{\complclass{L}}
\NewDocumentCommand{\NLogSpace}		{}{\complclass{NL}}
\NewDocumentCommand{\PTime}			{}{\complclass{P}}
\NewDocumentCommand{\NP}			{}{\complclass{NP}}
\NewDocumentCommand{\coNP}			{}{\complclass{coNP}}
\NewDocumentCommand{\PH}			{}{\complclass{PH}}
\NewDocumentCommand{\PSpace}		{}{\complclass{PSpace}}
\NewDocumentCommand{\NPSpace}		{}{\complclass{NPSpace}}
\NewDocumentCommand{\ExpTime}		{}{\complclass{ExpTime}}
\NewDocumentCommand{\NExpTime}		{}{\complclass{NExpTime}}
\NewDocumentCommand{\ExpSpace}		{}{\complclass{ExpSpace}}
\NewDocumentCommand{\TwoExpTime}	{}{\complclass{2ExpTime}}
\NewDocumentCommand{\NTwoExpTime}	{}{\complclass{N2ExpTime}}
\NewDocumentCommand{\TwoExpSpace}	{}{\complclass{2ExpSpace}}

% For tripple- and higher-exponential classes, use these.
% If the optional argument is specified, typset e.g. ‘5ExpTime’ using the small caps font.
% Otherwise, produce e.g. ‘kExpTime’, where the ‘k’ is typeset in the default math font.
% The latter is occasionally useful when talking about all higher exponential classes.
\NewDocumentCommand{\kExpTime}{ o }{
	\IfValueTF {#1} {
		\complclass{#1ExpTime}
	} {
		\ensuremath{k\mathsc{ExpTime}}\xspace
	}
}
\NewDocumentCommand{\NkExpTime}{ o }{
	\IfValueTF {#1} {
		\complclass{N#1ExpTime}
	} {
		\ensuremath{\mathsc{N}k\mathsc{ExpTime}}\xspace
	}
}
\NewDocumentCommand{\kExpSpace}{ o }{
	\IfValueTF {#1} {
		\complclass{#1ExpSpace}
	} {
		\ensuremath{k}\complclass{ExpSpace}
	}
}

% File has been loaded.
\toggletrue{25-complexity-classes}
